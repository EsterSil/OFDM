\documentclass[a4paper,12pt]{article} 

\usepackage{cmap}					% поиск в PDF
\usepackage{mathtext} 				% русские буквы в фомулах
\usepackage[T2A]{fontenc}			% кодировка
\usepackage[utf8]{inputenc}			% кодировка исходного текста
\usepackage[english,russian]{babel}	% локализация и переносы

%%% Дополнительная работа с математикой
\usepackage{amsmath,amsfonts,amssymb,amsthm,mathtools} % AMS
\usepackage{icomma} % "Умная" запятая: $0,2$ --- число, $0, 2$ --- перечисление
\usepackage[left=2cm,right=2cm,
    top=2cm,bottom=2cm,bindingoffset=0cm]{geometry}
\usepackage{indentfirst}
\usepackage{algorithm}
\usepackage[]{algpseudocode}
%\usepackage{algorithm2e}

%% Номера формул
\mathtoolsset{showonlyrefs=true} % Показывать номера только у тех формул, на которые есть \eqref{} в тексте.

%% Шрифты
\usepackage{euscript}	 % Шрифт Евклид
\usepackage{mathrsfs} % Красивый матшрифт
\usepackage{multirow}

%% Свои команды
%\DeclareMathOperator{\sgn}{\mathop{sgn}}

%% Перенос знаков в формулах (по Львовскому)
\newcommand*{\hm}[1]{#1\nobreak\discretionary{}
{\hbox{$\mathsurround=0pt #1$}}{}}

%%% Заголовок

\usepackage{graphicx}
\begin{document} % конец преамбулы, начало документа
\begin{titlepage}
  \begin{center}
    \large
    МОСКОВСКИЙ ФИЗИКО-ТЕХНИЧЕСКИЙ ИНСТИТУТ\\ (ГОСУДАРСТВЕННЫЙ УНИВЕРСИТЕТ)\\[60mm]
   
    
 
    Кафедра мультимедийных технологий и телекоммуникаций\\[5mm]
 
    \textsc{ЛАБОРАТОРНАЯ РАБОТА \\ по теме  }\\[5mm]
     
    {\LARGE OFDM}\\[6mm]
     \textsc{Методические указания }\\[100mm]
  \bigskip
     
      
\begin{center}

  под редакцией (*)

\end{center}
    
\end{center}

 


 
\begin{center}

 27.12.2017

\end{center}
\end{titlepage}

\section{Теоретические основы}
\section{Модулятор}
\section{Канал}

\section{Демодулятор}

\section{Дополнения}
\section{Практические задания}
\begin{abstract}
     Студентам предлагается написать программу, имитирующую преобразования и процессы, происходящие с сигналом, на всех трех этапах его существования (в рамках OFDM- модуляции): образования, передачи и приема.
     Выполнять задание можно на любом удобном языке програмирования (приоритетные языки matlab, python). 
     Данная глава содержит последовательное описание каждого этапа построения модели и ожидаемые промежуточные результаты.
    
\end{abstract}

Разработка модели состоит из пяти этапов, на каждом из которых сизнал проходит полный 'жизненный цикл', то есть кодирование бит, формирование сигнала, передачу в канал, прием и декодирование.  
Каждый следующий этап добавляет новые элементы обработки сигнала, учитывающие реальные физические процессы. 

\subsection{Первый этап.}
Задачи: 

*формирование битовой последовательности

*отображение групп битов на комплексную плоскость

*формирование OFDM-символа

*цап символа

*передача символьной последовательности в канал

*разбиение на символы на принимающей стороне

*ацп символа

*выделение информационной полосы частот

*отображение комплексных чисел в двоичный битовый вид

* контроль принятой информации



\subsubsection{Формирование битовой последовательности}



\end{document}
