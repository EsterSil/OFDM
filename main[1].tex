\documentclass[a4paper,12pt]{article} 

\usepackage{cmap}					% поиск в PDF
\usepackage{mathtext} 				% русские буквы в фомулах
\usepackage[T2A]{fontenc}			% кодировка
\usepackage[utf8]{inputenc}			% кодировка исходного текста
\usepackage[english,russian]{babel}	% локализация и переносы

%%% Дополнительная работа с математикой
\usepackage{amsmath,amsfonts,amssymb,amsthm,mathtools} % AMS
\usepackage{icomma} % "Умная" запятая: $0,2$ --- число, $0, 2$ --- перечисление
\usepackage[left=2cm,right=2cm,
    top=2cm,bottom=2cm,bindingoffset=0cm]{geometry}
\usepackage{indentfirst}
\usepackage{algorithm}
\usepackage[]{algpseudocode}
%\usepackage{algorithm2e}

%% Номера формул
\mathtoolsset{showonlyrefs=true} % Показывать номера только у тех формул, на которые есть \eqref{} в тексте.

%% Шрифты
\usepackage{euscript}	 % Шрифт Евклид
\usepackage{mathrsfs} % Красивый матшрифт
\usepackage{multirow}

%% Свои команды
%\DeclareMathOperator{\sgn}{\mathop{sgn}}

%% Перенос знаков в формулах (по Львовскому)
\newcommand*{\hm}[1]{#1\nobreak\discretionary{}
{\hbox{$\mathsurround=0pt #1$}}{}}

%%% Заголовок

\usepackage{graphicx}


\begin{document} % конец преамбулы, начало документа
\begin{titlepage}
  \begin{center}
    \large
    МОСКОВСКИЙ ФИЗИКО-ТЕХНИЧЕСКИЙ ИНСТИТУТ\\ (ГОСУДАРСТВЕННЫЙ УНИВЕРСИТЕТ)\\[60mm]
   
    
 
    Кафедра мультимедийных технологий и телекоммуникаций\\[5mm]
 
    \textsc{ЛАБОРАТОРНАЯ РАБОТА \\ по теме  }\\[5mm]
     
    {\LARGE OFDM}\\[6mm]
     \textsc{Методические указания }\\[100mm]
  \bigskip
     
      
\begin{center}

  под редакцией (*)

\end{center}
    
\end{center}

 


 
\begin{center}

 27.12.2017

\end{center}
\end{titlepage}

\section{Теоретические основы}
\section{Модулятор}
\section{Канал}

\section{Демодулятор}

\section{Дополнения}
\section{Практические задания}
\subsection*{Постановка задачи}
     Студентам предлагается написать программу, имитирующую преобразования и процессы, происходящие с сигналом в рамках OFDM- модуляции, на всех трех этапах его существования: образования, передачи и приема.
     Выполнять задание можно на любом удобном языке програмирования (приоритетные языки matlab, python). 
     Данная глава содержит последовательное описание каждого этапа и ожидаемые промежуточные результаты.
    

Разработка модели состоит из пяти этапов, на каждом из которых сигнал проходит полный 'жизненный цикл', то есть кодирование бит, формирование сигнала, передачу в канал, прием и декодирование.  
На каждом этапе нам предстоит добавлять новые элементы обработки сигнала, учитывающие реальные физические процессы. 

\subsection{Первый этап. Простейший цикл передачи-приема}
Требуется сформировать простейший 'жизненный цикл' сигнала.
Считать, что канал идеальный и не вносит искажений.
Последовательность преобразований отображена на Рис.\ref{fg:schem1}) 
\begin{figure}[h!]
\centering
\includegraphics[width=1\textwidth]{schem1.jpg}
\caption{Исследуемый кластер} \label{fg:schem1}
\end{figure}

Рекомендуется осуществлять кадждый этап преобразования в виде отдельной функции, что бы отслеживать результаты. 

Нам потребуются слудующие входные параметры модели:
* n\underline{ }carriers - число несущих сигнала (по дефолту 400).  %ссылка на параграф
* n\underline{ }fft - порядок преобразования фурье (1024 или 2048) (как влияет на сигнал изменение числа несущих или числа отсчетов в FFT?)%ссылка на параграф
* constellation - параметр, задающий модуляцию. В простейшем случае BPSK, либо QPSK, либо 16-QAM %ссылка на параграф

Эти переменные считаются параметрами приемо-передающей системы,  известными как на передатчике, так и на приемнике.

\subsubsection{Передатчик}
\subsubsection*{Формирование битовой последовательности}

Входную битовую последовательность можно сформировать случайным образом или импортировать из реального файла.
Второе сделать предпочтительнее, так как данные в изображениях сильно кореллированны. Это позволит пронаблюдать характерные особенности сигнала, например, периодичности,  соответсвующие блокам изображения с одинаковыми цветами.  %вставить рисунок сигнала пик фактора для битов из картинки

Результатом выполнения функции должна быть некоторая последовательность нулей и единиц:

$ 100101111110010000 ...$





\subsubsection*{Маппер}
Маппер - функция, которая осуществляет отображение групп битов на комплексную плоскость. 
В пункте %ссылка%
описаны различные варианты модуляций. 
Предлагается выбирать тип в зависимости от параметра constellation.
Данный параметр определяет разрядность одного отсчета (например, для кодирования всех точек 16 QAM требуется 4 бита, для QPSK  - 2 бита и,соответсвенно, для BPSK - 1) (см. Рис.\ref{fg:constellations}).

\begin{figure}[H]
\centering
\includegraphics[width=1\textwidth]{constellations}
\caption{Типы созвездий:\\ a) 16-QAM;\\b)QPSK;\\c)BPSK} \label{fg:constellations}
\end{figure}

При составлении созвездия необходимо использовать коды Грея, что бы минимизировать ошибку при неверном распознавании. % ссылка
Входной поток битов делится на 4ки или 2ки бит, которым в соответсвие ставится точка из созвездия:

\begin{figure}[H]
\centering
\includegraphics[width=1\textwidth]{schem2}
\end{figure}
 
 На выходе функции остается последовательность точек комплексной плоскости. 





\subsubsection*{Формирование OFDM-символа}

OFDM-модуляция предполагает частотное разделение сигнала. 
Передача информации в канале осуществляется пачками (они же OFDM-символы). 
В полосе выделяется  \textit{n\underline{ }carriers} поднесущих частоты, каждая из которых за время передачи одного символа несет информацию об одной информационной точке.
Следовательно, возможна передача не более  \textit{n\underline{ }carriers} точек за раз. 

На практике информационный объем одной передачи снижается, например, при выделении пилотных несущих (см последний этап), или при резервировании тона для снижения пик-фактора. 
Таким образом, целью данной функции является выделение необходимого числа точек из потока, а так же их подготовка к преобразованию Фурье. 
 
 \subsubsection*{Преобразование фурье}
 
Для передачи в настоящий физический канал сигнал требуется преобразовать в вид, доступный для передачи. 
А именно в аналоговую фнкцию, непрерывную по времени $S = S(t)$. 
По сути, совершить цифро-аналоговое преобразование. 

Поскольку мы моделируем цифровой сигнал, а не аналоговый 
воспользуемся теоремой Котельникова  (ссылка)
, которая позволяет рассматривать сигнал не как континуальный, а как набор дискретных отсчетов. 
Пока что наш сигнал предствляет собой последовательность комплексных точек. 
Их можно считать частотными отсчетами, так как мы предполагаем, что каждая поднесущая несет информацию (фазу и амплитуду) об отдельной точке.  
Представить частотные отчсчеты в виде функции от времени нам поможет преобразование Фурье.

Большинство современных языков программирования содержат готовые пакеты с преобразованием Фурье (алгоритм  Fast Fourier Transform и обратный ему), так что не будем изобретать велосипед и воспользуемся ими.
Тем более, что даный алгоритм достаточно сложен для понимания и потребует много времени. 

FFT принмает на вход последовательность дискретных временных отсчетов сигнала и возвращает дискретную последовательность отсчетов частотных. iFFT,  соответсвенно, производит обратную операцию. 
Поэтому на текущем этапе обработки сигнала нас интересует именно обратное преобразование, поскольку дальше сигнал отправится в канал
(см. Рис.\ref{fg:fft})


\begin{figure}[h!]
\centering
\includegraphics[]{FFT.png}
\caption{FFT} \label{fg:fft}
\end{figure}

Перед непоследственно самим преобразованием необходимо дополнить последовательность отсчетов нулями от значения n\underline{ }carriers до n\underline{ }fft (операция upsempling)
Делается это по нескольким причинам.
Прежде всего, исходят из особенностей алгоритма: преобразования на $2^n$ отсчетов проиходят существенно быстрее. 
Кроме того, upsempling позволяет существенно уплотнить временные отсчеты, так как имея всего n\underline{ }carriers частотных отсчетов, мы получим целых n\underline{ }fft временных. 
Сравните плотность отсчетов на Рис.\ref{fg:fft400} a) и b).

\begin{figure}[H]
\begin{minipage}[h]{\linewidth}
\center{\includegraphics[width=0.5\linewidth]{fft400} \\ a)}
\end{minipage}

\begin{minipage}[h]{\linewidth}
\center{\includegraphics[width=0.5\linewidth]{fft1024} \\ b)}
\end{minipage}
\caption{Часть сигнала для a)400 отсчетов и b)1024 отсчетов.}
\label{fg:fft400}
\end{figure}




Выходом данного перобразования будет последовательность временных отсчетов сигнала.

 \subsubsection*{Передача символьной последовательности в канал}
 
Для каждого OFDM-символа формируется косочек сигнала, которые необходимо "сшить" вместе перед передачей. 
Позднее здесь будет добавлен защитный интервал, сейчас можно просто склеить результаты iFFT-преобразования.

\subsubsection{Канал}

Канал считается идеальным, так что на этом этапе он не моделируется. 


\subsubsection{Приемник}

 \subsubsection*{Разбиение на символы на принимающей стороне}
 Все дальнейшие преобразования происходят на приемнике.
 Пока что не требуется проводить синхронизацию, так как канал идеален, и прием начинается с первого отсчета первого символа. 
 Аналогично как и в передатчике, следует разбить принятый сигнал на отдельные OFDM-символы для дальнейшего перобразования Фурье. 
 
 
\subsubsection*{Преобразование Фурье}
Необходимо выполнить аналого цифровое преобразование символа,  что бы получть значения отдельных частотных отсчетов.
Для этого применяем брпрямое преобразование Фурье (FFT).
На выходе получим набор точек комплексной плоскости.

\subsubsection*{Выделение информационной полосы частот}
Поскольку мы дополнили сигнал нулями, необходимо отбросить лишние частотные составляющие, которые не несут никакой информации. 
На текущем этапе их можно просто отбросить.
на выходе должны получить  n\underline{ }carriers точек.  


*отображение комплексных чисел в двоичный битовый вид

* контроль принятой информации







\end{document}
